% Options for packages loaded elsewhere
\PassOptionsToPackage{unicode}{hyperref}
\PassOptionsToPackage{hyphens}{url}
%
\documentclass[
]{article}
\usepackage{amsmath,amssymb}
\usepackage{iftex}
\ifPDFTeX
  \usepackage[T1]{fontenc}
  \usepackage[utf8]{inputenc}
  \usepackage{textcomp} % provide euro and other symbols
\else % if luatex or xetex
  \usepackage{unicode-math} % this also loads fontspec
  \defaultfontfeatures{Scale=MatchLowercase}
  \defaultfontfeatures[\rmfamily]{Ligatures=TeX,Scale=1}
\fi
\usepackage{lmodern}
\ifPDFTeX\else
  % xetex/luatex font selection
\fi
% Use upquote if available, for straight quotes in verbatim environments
\IfFileExists{upquote.sty}{\usepackage{upquote}}{}
\IfFileExists{microtype.sty}{% use microtype if available
  \usepackage[]{microtype}
  \UseMicrotypeSet[protrusion]{basicmath} % disable protrusion for tt fonts
}{}
\makeatletter
\@ifundefined{KOMAClassName}{% if non-KOMA class
  \IfFileExists{parskip.sty}{%
    \usepackage{parskip}
  }{% else
    \setlength{\parindent}{0pt}
    \setlength{\parskip}{6pt plus 2pt minus 1pt}}
}{% if KOMA class
  \KOMAoptions{parskip=half}}
\makeatother
\usepackage{xcolor}
\usepackage[margin=1in]{geometry}
\usepackage{color}
\usepackage{fancyvrb}
\newcommand{\VerbBar}{|}
\newcommand{\VERB}{\Verb[commandchars=\\\{\}]}
\DefineVerbatimEnvironment{Highlighting}{Verbatim}{commandchars=\\\{\}}
% Add ',fontsize=\small' for more characters per line
\usepackage{framed}
\definecolor{shadecolor}{RGB}{248,248,248}
\newenvironment{Shaded}{\begin{snugshade}}{\end{snugshade}}
\newcommand{\AlertTok}[1]{\textcolor[rgb]{0.94,0.16,0.16}{#1}}
\newcommand{\AnnotationTok}[1]{\textcolor[rgb]{0.56,0.35,0.01}{\textbf{\textit{#1}}}}
\newcommand{\AttributeTok}[1]{\textcolor[rgb]{0.13,0.29,0.53}{#1}}
\newcommand{\BaseNTok}[1]{\textcolor[rgb]{0.00,0.00,0.81}{#1}}
\newcommand{\BuiltInTok}[1]{#1}
\newcommand{\CharTok}[1]{\textcolor[rgb]{0.31,0.60,0.02}{#1}}
\newcommand{\CommentTok}[1]{\textcolor[rgb]{0.56,0.35,0.01}{\textit{#1}}}
\newcommand{\CommentVarTok}[1]{\textcolor[rgb]{0.56,0.35,0.01}{\textbf{\textit{#1}}}}
\newcommand{\ConstantTok}[1]{\textcolor[rgb]{0.56,0.35,0.01}{#1}}
\newcommand{\ControlFlowTok}[1]{\textcolor[rgb]{0.13,0.29,0.53}{\textbf{#1}}}
\newcommand{\DataTypeTok}[1]{\textcolor[rgb]{0.13,0.29,0.53}{#1}}
\newcommand{\DecValTok}[1]{\textcolor[rgb]{0.00,0.00,0.81}{#1}}
\newcommand{\DocumentationTok}[1]{\textcolor[rgb]{0.56,0.35,0.01}{\textbf{\textit{#1}}}}
\newcommand{\ErrorTok}[1]{\textcolor[rgb]{0.64,0.00,0.00}{\textbf{#1}}}
\newcommand{\ExtensionTok}[1]{#1}
\newcommand{\FloatTok}[1]{\textcolor[rgb]{0.00,0.00,0.81}{#1}}
\newcommand{\FunctionTok}[1]{\textcolor[rgb]{0.13,0.29,0.53}{\textbf{#1}}}
\newcommand{\ImportTok}[1]{#1}
\newcommand{\InformationTok}[1]{\textcolor[rgb]{0.56,0.35,0.01}{\textbf{\textit{#1}}}}
\newcommand{\KeywordTok}[1]{\textcolor[rgb]{0.13,0.29,0.53}{\textbf{#1}}}
\newcommand{\NormalTok}[1]{#1}
\newcommand{\OperatorTok}[1]{\textcolor[rgb]{0.81,0.36,0.00}{\textbf{#1}}}
\newcommand{\OtherTok}[1]{\textcolor[rgb]{0.56,0.35,0.01}{#1}}
\newcommand{\PreprocessorTok}[1]{\textcolor[rgb]{0.56,0.35,0.01}{\textit{#1}}}
\newcommand{\RegionMarkerTok}[1]{#1}
\newcommand{\SpecialCharTok}[1]{\textcolor[rgb]{0.81,0.36,0.00}{\textbf{#1}}}
\newcommand{\SpecialStringTok}[1]{\textcolor[rgb]{0.31,0.60,0.02}{#1}}
\newcommand{\StringTok}[1]{\textcolor[rgb]{0.31,0.60,0.02}{#1}}
\newcommand{\VariableTok}[1]{\textcolor[rgb]{0.00,0.00,0.00}{#1}}
\newcommand{\VerbatimStringTok}[1]{\textcolor[rgb]{0.31,0.60,0.02}{#1}}
\newcommand{\WarningTok}[1]{\textcolor[rgb]{0.56,0.35,0.01}{\textbf{\textit{#1}}}}
\usepackage{graphicx}
\makeatletter
\def\maxwidth{\ifdim\Gin@nat@width>\linewidth\linewidth\else\Gin@nat@width\fi}
\def\maxheight{\ifdim\Gin@nat@height>\textheight\textheight\else\Gin@nat@height\fi}
\makeatother
% Scale images if necessary, so that they will not overflow the page
% margins by default, and it is still possible to overwrite the defaults
% using explicit options in \includegraphics[width, height, ...]{}
\setkeys{Gin}{width=\maxwidth,height=\maxheight,keepaspectratio}
% Set default figure placement to htbp
\makeatletter
\def\fps@figure{htbp}
\makeatother
\setlength{\emergencystretch}{3em} % prevent overfull lines
\providecommand{\tightlist}{%
  \setlength{\itemsep}{0pt}\setlength{\parskip}{0pt}}
\setcounter{secnumdepth}{-\maxdimen} % remove section numbering
% definitions for citeproc citations
\NewDocumentCommand\citeproctext{}{}
\NewDocumentCommand\citeproc{mm}{%
  \begingroup\def\citeproctext{#2}\cite{#1}\endgroup}
\makeatletter
 % allow citations to break across lines
 \let\@cite@ofmt\@firstofone
 % avoid brackets around text for \cite:
 \def\@biblabel#1{}
 \def\@cite#1#2{{#1\if@tempswa , #2\fi}}
\makeatother
\newlength{\cslhangindent}
\setlength{\cslhangindent}{1.5em}
\newlength{\csllabelwidth}
\setlength{\csllabelwidth}{3em}
\newenvironment{CSLReferences}[2] % #1 hanging-indent, #2 entry-spacing
 {\begin{list}{}{%
  \setlength{\itemindent}{0pt}
  \setlength{\leftmargin}{0pt}
  \setlength{\parsep}{0pt}
  % turn on hanging indent if param 1 is 1
  \ifodd #1
   \setlength{\leftmargin}{\cslhangindent}
   \setlength{\itemindent}{-1\cslhangindent}
  \fi
  % set entry spacing
  \setlength{\itemsep}{#2\baselineskip}}}
 {\end{list}}
\usepackage{calc}
\newcommand{\CSLBlock}[1]{\hfill\break\parbox[t]{\linewidth}{\strut\ignorespaces#1\strut}}
\newcommand{\CSLLeftMargin}[1]{\parbox[t]{\csllabelwidth}{\strut#1\strut}}
\newcommand{\CSLRightInline}[1]{\parbox[t]{\linewidth - \csllabelwidth}{\strut#1\strut}}
\newcommand{\CSLIndent}[1]{\hspace{\cslhangindent}#1}
\ifLuaTeX
  \usepackage{selnolig}  % disable illegal ligatures
\fi
\usepackage{bookmark}
\IfFileExists{xurl.sty}{\usepackage{xurl}}{} % add URL line breaks if available
\urlstyle{same}
\hypersetup{
  pdftitle={Sanchez-Sierra-David-PEC1},
  pdfauthor={David Sánchez Sierra},
  hidelinks,
  pdfcreator={LaTeX via pandoc}}

\title{Sanchez-Sierra-David-PEC1}
\author{David Sánchez Sierra}
\date{2025-04-01}

\begin{document}
\maketitle

\section{Análisis de datos ómicos (M0-157). Prueba de evaluación
continua.}\label{anuxe1lisis-de-datos-uxf3micos-m0-157.-prueba-de-evaluaciuxf3n-continua.}

\subsection{Tabla de contenidos}\label{tabla-de-contenidos}

\begin{enumerate}
\def\labelenumi{\arabic{enumi}.}
\tightlist
\item
  \hyperref[resumen]{Resumen}\\
\item
  \hyperref[objetivos]{Objetivos}\\
\item
  \hyperref[enfoque-y-muxe9todo-seguido]{Enfoque y método seguido} 3.1.
  \hyperref[selecciuxf3n-y-descarga-del-dataset]{Selección y descarga
  del dataset}\\
  3.2. \hyperref[creaciuxf3n-del-objeto-summarizedexperiment]{Creación
  del objeto SummarizedExperiment}\\
\item
  \hyperref[resultados]{Resultados}\\
  4.1.
  \hyperref[diferencias-entre-summarizedexperiment-y-expressionset]{Diferencias
  entre SummarizedExperiment y ExpressionSet}\\
  4.2. \hyperref[anuxe1lisis-exploratorio]{Análisis exploratorio}\\
  4.3. \hyperref[anova]{Análisis de metabolitos asociados a la pérdida
  muscular}\\
\item
  \hyperref[discusiuxf3n]{Discusión}\\
\item
  \hyperref[conclusiones]{Conclusiones}\\
\item
  \hyperref[bibliografuxeda]{Bibliografía}
\end{enumerate}

\subsection{1. Resumen}\label{resumen}

En esta actividad, se ha estudiado la relación entre metabolitos
sanguíneos y la condición de caquexia en pacientes con cáncer mediante
el uso de la clase SummarizedExperiment (SE). Se utilizó un conjunto de
datos que incluyó 63 metabolitos y 77 muestras de pacientes, los cuales
fueron analizados para identificar posibles biomarcadores relacionados
con la pérdida muscular. La creación del objeto (SE) se realizó con el
objetivo de aprender a organizar y estructurar unos datos ómicos de
forma adecuada para su correcto procesamiento. El análisis de dicha
clase verificó que su estructura fuera correcta, de forma que los
metabolitos quedaran almacenados en matrices de expresión y los
metadatos de las muestras (condición de caquexia) integrados en un solo
objeto. A través de un análisis de varianza (ANOVA), se identificaron 40
metabolitos significativamente asociados con la caquexia, de los cuales
7 mostraron una relación altamente significativa (p \textless{} 0.001).
Entre ellos, se destacaron metabolitos como creatinina, dimethylamina,
leucina, N.N.Dimethylglycina, pyroglutamata, quinolinata y valina,
previamente asociados con la caquexia en estudios previos. Estos
hallazgos sugieren que estos metabolitos pueden servir como
biomarcadores clave en el diagnóstico y seguimiento de la caquexia, lo
que subraya la importancia de la metabolómica en la investigación del
cáncer y sus complicaciones asociadas.

\subsection{2. Objetivos}\label{objetivos}

El objetivo principal de este estudio fue la creación, exploración y
análisis de datos utilizando el objeto SummarizedExperiment (SE), con el
fin de investigar las diferencias en los metabolitos entre pacientes con
y sin caquexia. Este enfoque se orientó hacia la organización y análisis
de datos ómicos, en particular datos de metabolómica, asociados con los
metadatos clínicos de los pacientes. Los objetivos específicos fueron:

\begin{enumerate}
\def\labelenumi{\arabic{enumi}.}
\item
  Crear el objeto SummarizedExperiment: Organizar los datos de expresión
  de metabolitos (63 metabolitos) y los metadatos asociados a las
  muestras de 77 pacientes en un solo objeto para facilitar su manejo y
  análisis.
\item
  Explorar la estructura del objeto SummarizedExperiment: Inspeccionar
  las dimensiones y los componentes clave del SE, como las matrices de
  expresión de metabolitos (almacenadas en el componente assays) y los
  metadatos (almacenados en colData), con el fin de verificar la
  correcta organización de los datos.
\item
  Realizar un análisis estadístico (ANOVA): Evaluar la relación entre
  los metabolitos y la condición de pérdida muscular en los pacientes
  (caquexia) mediante un análisis de varianza (ANOVA), con el fin de
  identificar metabolitos potencialmente significativos asociados con
  esta condición.
\end{enumerate}

\subsection{3. Enfoque y método
seguido}\label{enfoque-y-muxe9todo-seguido}

\subsubsection{3.1. Selección y descarga del
dataset}\label{selecciuxf3n-y-descarga-del-dataset}

Para esta actividad, se seleccionó el dataset ``caquexia'' obtenido del
repositorio de github adjunto al encunciado. Este conjunto de datos ha
sido ampliamente utilizado en diversos tutoriales de MetaboAnalyst, lo
que permitió expandir el análisis realizado al disponer de información
sobre numerosos enfoques y metodologías establecidas.

Las técnicas metabolómicas se han utilizado para estudiar los cambios
metabólicos, incluyendo las variaciones en las concentraciones de
metabolitos y las vías metabólicas alteradas en la progresión de la
caquexia relacionada con el cáncer (CC), así como para ampliar la
comprensión fundamental de la pérdida muscular(Cui et al. 2022). En este
dataset, se incluyen datos sobre 77 pacientes oncológicos, clasificados
en dos grupos según su estado de pérdida muscular: controles y
caquéxicos.

La descarga de los datos se realizó utilizando R, empleando la función
read.csv() para cargar el archivo en formato tabular:

\begin{Shaded}
\begin{Highlighting}[]
\NormalTok{dataset }\OtherTok{\textless{}{-}} \FunctionTok{read.csv}\NormalTok{(}\StringTok{"human\_cachexia.csv"}\NormalTok{, }\AttributeTok{header =} \ConstantTok{TRUE}\NormalTok{, }\AttributeTok{sep =} \StringTok{","}\NormalTok{, }\AttributeTok{stringsAsFactors =} \ConstantTok{FALSE}\NormalTok{)}
\FunctionTok{str}\NormalTok{(dataset, }\AttributeTok{max.level =} \DecValTok{0}\NormalTok{)}
\end{Highlighting}
\end{Shaded}

\begin{verbatim}
## 'data.frame':    77 obs. of  65 variables:
\end{verbatim}

Como puede observarse, el dataset contiene información sobre 65
variables, incluyendo las concentraciones de diversos metabolitos en las
muestras de orina asociadas con los dos grupos de pacientes. Además, el
conjunto de datos incluye el identificador de cada paciente y los
metadatos relacionados con las características de estos, en este caso el
tipo categórico (control o caquéxicos).

\subsubsection{3.2. Creación del objeto
SummarizedExperiment}\label{creaciuxf3n-del-objeto-summarizedexperiment}

Tras cargar los datos, se creó un objeto de clase SummarizedExperiment
proporcionada por el paquete Bioconductor, el cual permite estructurar
los datos y metadatos asociados a cada observación, facilitando su
manipulación y análisis. Primero se asignaron los valores de Patient.ID
como nombres de fila para el dataset, asegurando que cada muestra
tuviera una identificación única en la matriz. Tras ello, se eliminó la
columna Patient.ID del conjunto de datos, ya que solo se necesitaba para
asignar los nombres de las filas, y no debía formar parte de los datos
de expresión.

A continuación, se extrajeron las mediciones de los metabolitos del
dataset, excluyendo la columna Muscle.loss, que corresponde a la
información de los metadatos y no a las concentraciones de los
metabolitos. Las concentraciones de metabolitos se almacenaron en una
matriz de expresión (expr\_matrix), la cual fue transpuesta para que las
muestras fueran representadas en las columnas, lo que facilita la
asociación con los metadatos. En el siguiente paso, se creó el DataFrame
para los metadatos que debía contener la columna Muscle.loss
(información sobre el estado de caquexia de las muestras).

El código que implementa este procedimiento es el siguiente:

\begin{Shaded}
\begin{Highlighting}[]
\CommentTok{\# Cargar paquetes necesarios}
\FunctionTok{library}\NormalTok{(S4Vectors)}
\FunctionTok{library}\NormalTok{(SummarizedExperiment)}
\end{Highlighting}
\end{Shaded}

\begin{Shaded}
\begin{Highlighting}[]
\CommentTok{\# Asignar ID como nombre de fila}
\FunctionTok{rownames}\NormalTok{(dataset) }\OtherTok{\textless{}{-}}\NormalTok{ dataset}\SpecialCharTok{$}\NormalTok{Patient.ID  }
\NormalTok{dataset }\OtherTok{\textless{}{-}}\NormalTok{ dataset[, }\SpecialCharTok{{-}}\DecValTok{1}\NormalTok{] }\CommentTok{\# Eliminar la columna \textquotesingle{}Patient ID\textquotesingle{}}

\CommentTok{\# Asegurarse de que \textquotesingle{}Muscle.loss\textquotesingle{} es excluida correctamente}
\NormalTok{expr\_matrix }\OtherTok{\textless{}{-}} \FunctionTok{as.matrix}\NormalTok{(dataset[, }\SpecialCharTok{{-}}\FunctionTok{which}\NormalTok{(}\FunctionTok{names}\NormalTok{(dataset) }\SpecialCharTok{==} \StringTok{"Muscle.loss"}\NormalTok{)])}

\CommentTok{\# Transponer la matriz para que las muestras sean columnas}
\NormalTok{expr\_matrix }\OtherTok{\textless{}{-}} \FunctionTok{t}\NormalTok{(expr\_matrix)}

\CommentTok{\# Crear el dataframe de metadatos (solo \textquotesingle{}Muscle.loss\textquotesingle{})}
\NormalTok{metadata\_df }\OtherTok{\textless{}{-}} \FunctionTok{DataFrame}\NormalTok{(}\AttributeTok{Muscle.loss =}\NormalTok{ dataset}\SpecialCharTok{$}\NormalTok{Muscle.loss)}
\FunctionTok{rownames}\NormalTok{(metadata\_df) }\OtherTok{\textless{}{-}} \FunctionTok{rownames}\NormalTok{(dataset)  }\CommentTok{\# Asegurar nombres de fila en metadatos}

\CommentTok{\# Crear el objeto SummarizedExperiment}
\NormalTok{se\_object }\OtherTok{\textless{}{-}} \FunctionTok{SummarizedExperiment}\NormalTok{(}
  \AttributeTok{assays =} \FunctionTok{list}\NormalTok{(}\AttributeTok{counts =}\NormalTok{ expr\_matrix),}
  \AttributeTok{colData =}\NormalTok{ metadata\_df}
\NormalTok{)}

\CommentTok{\# Verificar el objeto creado}
\NormalTok{se\_object}
\end{Highlighting}
\end{Shaded}

\begin{verbatim}
## class: SummarizedExperiment 
## dim: 63 77 
## metadata(0):
## assays(1): counts
## rownames(63): X1.6.Anhydro.beta.D.glucose X1.Methylnicotinamide ...
##   pi.Methylhistidine tau.Methylhistidine
## rowData names(0):
## colnames(77): PIF_178 PIF_087 ... NETL_003_V1 NETL_003_V2
## colData names(1): Muscle.loss
\end{verbatim}

\begin{Shaded}
\begin{Highlighting}[]
\CommentTok{\# Guardar el objeto SummarizedExperiment en un archivo .Rda}
\FunctionTok{save}\NormalTok{(se\_object, }\AttributeTok{file =} \StringTok{"SE\_object.Rda"}\NormalTok{)}
\end{Highlighting}
\end{Shaded}

\subsection{4. Resultados}\label{resultados}

\subsubsection{4.1. Diferencias entre SummarizedExperiment y
ExpressionSet}\label{diferencias-entre-summarizedexperiment-y-expressionset}

La clase SummarizedExperiment es una extensión moderna de ExpressionSet,
con varias mejoras en cuanto a la estructura de almacenamiento y la
interoperabilidad con otras herramientas de Bioconductor. Una de las
diferencias más notables es que SummarizedExperiment utiliza el
componente assays para manejar las matrices de datos en lugar de la
función exprs() que se usaba en ExpressionSet. Además,
SummarizedExperiment también incorpora colData y rowData para el
almacenamiento de los metadatos relativos a las muestras y las
características, respectivamente, sustituyendo a las funciones pData() y
fData() que se encontraban en ExpressionSet. Esta modificación
simplifica y vuelve más adaptable la gestión de dichos metadatos. Un
beneficio adicional significativo de SummarizedExperiment reside en su
mejorada interoperabilidad con análisis de datos multi-ómicos, lo que la
establece como una herramienta más idónea para la realización de
estudios que combinan distintos tipos de datos ómicos, tales como datos
genómicos, transcriptómicos y proteómicos ({``Convert {Seurat Object} to
{Summarised Experiment} or {ExpressionSet}''} n.d.).

Estas diferencias hacen que SummarizedExperiment resulte una opción más
adecuada para manejar y analizar datos provenientes de diversas fuentes
dentro de un mismo entorno de trabajo, otorgándole una mayor
flexibilidad en comparación con ExpressionSet.

\subsubsection{4.2. Análisis
exploratorio}\label{anuxe1lisis-exploratorio}

En esta sección se muestra el análisis inicial del objeto
SummarizedExperiment (SE), que contiene los datos de expresión de
metabolitos y los metadatos asociados a las muestras:

\begin{Shaded}
\begin{Highlighting}[]
\FunctionTok{dim}\NormalTok{(se\_object)}
\end{Highlighting}
\end{Shaded}

\begin{verbatim}
## [1] 63 77
\end{verbatim}

\begin{Shaded}
\begin{Highlighting}[]
\FunctionTok{head}\NormalTok{(}\FunctionTok{assay}\NormalTok{(se\_object)[ , }\DecValTok{1}\SpecialCharTok{:}\DecValTok{10}\NormalTok{])}
\end{Highlighting}
\end{Shaded}

\begin{verbatim}
##                             PIF_178 PIF_087 PIF_090 NETL_005_V1 PIF_115 PIF_110
## X1.6.Anhydro.beta.D.glucose   40.85   62.18  270.43      154.47   22.20  212.72
## X1.Methylnicotinamide         65.37  340.36   64.72       52.98   73.70   31.82
## X2.Aminobutyrate              18.73   24.29   12.18      172.43   15.64   18.36
## X2.Hydroxyisobutyrate         26.05   41.68   65.37       74.44   83.93   80.64
## X2.Oxoglutarate               71.52   67.36   23.81     1199.91   33.12   47.94
## X3.Aminoisobutyrate         1480.30  116.75   14.30      555.57   29.67   17.46
##                             NETL_019_V1 NETCR_014_V1 NETCR_014_V2 PIF_154
## X1.6.Anhydro.beta.D.glucose      151.41        31.50        51.42  117.92
## X1.Methylnicotinamide             36.60         6.82        30.27   52.46
## X2.Aminobutyrate                   8.67         4.18         7.54   19.49
## X2.Hydroxyisobutyrate             42.52        12.94        34.81   72.24
## X2.Oxoglutarate                  223.63        25.03        80.64   73.70
## X3.Aminoisobutyrate               56.26         8.67        17.99   57.97
\end{verbatim}

El SE tiene una dimensión de 63 filas y 77 columnas, lo que verifica que
contiene 63 metabolitos (características) y 77 pacientes (muestras),
confirmando una estructura típica de esta clase de objetos. Los datos de
expresión de metabolitos son accesibles mediante la función assay(), tal
y como muestra el bloque de código, en el que se visualizan los datos de
los primeras 10 pacientes para los 6 primeros metabolitos.

A continuación, se presenta el contenido del objeto colData, que
almacena los metadatos de las muestras. En este caso, contiene una sola
columna denominada Muscle.loss, que refleja la condición de cada
paciente (ya sea cachexic o control):

\begin{Shaded}
\begin{Highlighting}[]
\FunctionTok{colData}\NormalTok{(se\_object)}
\end{Highlighting}
\end{Shaded}

\begin{verbatim}
## DataFrame with 77 rows and 1 column
##              Muscle.loss
##              <character>
## PIF_178         cachexic
## PIF_087         cachexic
## PIF_090         cachexic
## NETL_005_V1     cachexic
## PIF_115         cachexic
## ...                  ...
## NETCR_019_V2     control
## NETL_012_V1      control
## NETL_012_V2      control
## NETL_003_V1      control
## NETL_003_V2      control
\end{verbatim}

Finalmente, al inspeccionar los primeros nombres de las columnas con
colnames(se\_object), se observan los identificadores de los pacientes,
como PIF\_178, PIF\_087, PIF\_090, entre otros.

\begin{Shaded}
\begin{Highlighting}[]
\FunctionTok{head}\NormalTok{(}\FunctionTok{colnames}\NormalTok{(se\_object))}
\end{Highlighting}
\end{Shaded}

\begin{verbatim}
## [1] "PIF_178"     "PIF_087"     "PIF_090"     "NETL_005_V1" "PIF_115"    
## [6] "PIF_110"
\end{verbatim}

Este análisis preliminar del SE nos proporciona una visión general de la
estructura de los datos y nos permite verificar la correcta organización
tanto de las características (metabolitos) como de las muestras
(pacientes).

\subsubsection{4.3. Anova}\label{anova}

\begin{Shaded}
\begin{Highlighting}[]
\CommentTok{\# Cargar las librerías necesarias}
\FunctionTok{library}\NormalTok{(ggplot2)}
\end{Highlighting}
\end{Shaded}

\begin{verbatim}
## Warning: package 'ggplot2' was built under R version 4.4.2
\end{verbatim}

\begin{Shaded}
\begin{Highlighting}[]
\FunctionTok{library}\NormalTok{(dplyr)}
\end{Highlighting}
\end{Shaded}

\begin{verbatim}
## Warning: package 'dplyr' was built under R version 4.4.2
\end{verbatim}

\begin{verbatim}
## 
## Adjuntando el paquete: 'dplyr'
\end{verbatim}

\begin{verbatim}
## The following object is masked from 'package:Biobase':
## 
##     combine
\end{verbatim}

\begin{verbatim}
## The following objects are masked from 'package:GenomicRanges':
## 
##     intersect, setdiff, union
\end{verbatim}

\begin{verbatim}
## The following object is masked from 'package:GenomeInfoDb':
## 
##     intersect
\end{verbatim}

\begin{verbatim}
## The following objects are masked from 'package:IRanges':
## 
##     collapse, desc, intersect, setdiff, slice, union
\end{verbatim}

\begin{verbatim}
## The following object is masked from 'package:matrixStats':
## 
##     count
\end{verbatim}

\begin{verbatim}
## The following objects are masked from 'package:S4Vectors':
## 
##     first, intersect, rename, setdiff, setequal, union
\end{verbatim}

\begin{verbatim}
## The following objects are masked from 'package:BiocGenerics':
## 
##     combine, intersect, setdiff, union
\end{verbatim}

\begin{verbatim}
## The following objects are masked from 'package:stats':
## 
##     filter, lag
\end{verbatim}

\begin{verbatim}
## The following objects are masked from 'package:base':
## 
##     intersect, setdiff, setequal, union
\end{verbatim}

\begin{Shaded}
\begin{Highlighting}[]
\FunctionTok{dim}\NormalTok{(}\FunctionTok{assay}\NormalTok{(se\_object))}
\end{Highlighting}
\end{Shaded}

\begin{verbatim}
## [1] 63 77
\end{verbatim}

\begin{Shaded}
\begin{Highlighting}[]
\NormalTok{anova\_results }\OtherTok{\textless{}{-}} \FunctionTok{apply}\NormalTok{(}\FunctionTok{assay}\NormalTok{(se\_object), }\DecValTok{1}\NormalTok{, }\ControlFlowTok{function}\NormalTok{(x) \{}
\NormalTok{  aov\_result }\OtherTok{\textless{}{-}} \FunctionTok{aov}\NormalTok{(x }\SpecialCharTok{\textasciitilde{}} \FunctionTok{colData}\NormalTok{(se\_object)}\SpecialCharTok{$}\NormalTok{Muscle.loss)}
  \CommentTok{\# Extraemos el valor p sin la estructura de lista}
\NormalTok{  p\_value }\OtherTok{\textless{}{-}} \FunctionTok{summary}\NormalTok{(aov\_result)[[}\DecValTok{1}\NormalTok{]][}\StringTok{"Pr(\textgreater{}F)"}\NormalTok{][}\DecValTok{1}\NormalTok{, }\DecValTok{1}\NormalTok{]}
  \FunctionTok{return}\NormalTok{(p\_value)}
\NormalTok{\})}

\CommentTok{\# Crear el DataFrame correctamente}
\NormalTok{results\_df }\OtherTok{\textless{}{-}} \FunctionTok{data.frame}\NormalTok{(}
  \AttributeTok{metabolite =} \FunctionTok{rownames}\NormalTok{(}\FunctionTok{assay}\NormalTok{(se\_object)),}
  \AttributeTok{p\_value =}\NormalTok{ anova\_results}
\NormalTok{)}

\CommentTok{\# Verificamos que el DataFrame se crea correctamente}
\NormalTok{results\_df}
\end{Highlighting}
\end{Shaded}

\begin{verbatim}
##                                              metabolite      p_value
## X1.6.Anhydro.beta.D.glucose X1.6.Anhydro.beta.D.glucose 0.0507913658
## X1.Methylnicotinamide             X1.Methylnicotinamide 0.9343185339
## X2.Aminobutyrate                       X2.Aminobutyrate 0.0274454761
## X2.Hydroxyisobutyrate             X2.Hydroxyisobutyrate 0.0052880207
## X2.Oxoglutarate                         X2.Oxoglutarate 0.2250725245
## X3.Aminoisobutyrate                 X3.Aminoisobutyrate 0.1779554046
## X3.Hydroxybutyrate                   X3.Hydroxybutyrate 0.0011853671
## X3.Hydroxyisovalerate             X3.Hydroxyisovalerate 0.0078274621
## X3.Indoxylsulfate                     X3.Indoxylsulfate 0.0089161341
## X4.Hydroxyphenylacetate         X4.Hydroxyphenylacetate 0.4818107935
## Acetate                                         Acetate 0.0060731953
## Acetone                                         Acetone 0.3723624590
## Adipate                                         Adipate 0.0274331625
## Alanine                                         Alanine 0.0011788609
## Asparagine                                   Asparagine 0.0067852184
## Betaine                                         Betaine 0.0029929659
## Carnitine                                     Carnitine 0.0621334639
## Citrate                                         Citrate 0.0128569793
## Creatine                                       Creatine 0.0526324364
## Creatinine                                   Creatinine 0.0005129808
## Dimethylamine                             Dimethylamine 0.0004460069
## Ethanolamine                               Ethanolamine 0.0265705687
## Formate                                         Formate 0.0174219763
## Fucose                                           Fucose 0.0058534857
## Fumarate                                       Fumarate 0.0590430806
## Glucose                                         Glucose 0.0333063584
## Glutamine                                     Glutamine 0.0010607678
## Glycine                                         Glycine 0.0281395160
## Glycolate                                     Glycolate 0.0563341550
## Guanidoacetate                           Guanidoacetate 0.1378998638
## Hippurate                                     Hippurate 0.0232194018
## Histidine                                     Histidine 0.0109704361
## Hypoxanthine                               Hypoxanthine 0.2555158507
## Isoleucine                                   Isoleucine 0.1326513615
## Lactate                                         Lactate 0.1228434803
## Leucine                                         Leucine 0.0002695046
## Lysine                                           Lysine 0.2817634974
## Methylamine                                 Methylamine 0.0019466201
## Methylguanidine                         Methylguanidine 0.2616448016
## N.N.Dimethylglycine                 N.N.Dimethylglycine 0.0001554346
## O.Acetylcarnitine                     O.Acetylcarnitine 0.0616844278
## Pantothenate                               Pantothenate 0.5353413018
## Pyroglutamate                             Pyroglutamate 0.0004845363
## Pyruvate                                       Pyruvate 0.0174471086
## Quinolinate                                 Quinolinate 0.0001185108
## Serine                                           Serine 0.0037584213
## Succinate                                     Succinate 0.0113771704
## Sucrose                                         Sucrose 0.1194324540
## Tartrate                                       Tartrate 0.4464672671
## Taurine                                         Taurine 0.0323497353
## Threonine                                     Threonine 0.0032881849
## Trigonelline                               Trigonelline 0.0128629342
## Trimethylamine.N.oxide           Trimethylamine.N.oxide 0.0415941735
## Tryptophan                                   Tryptophan 0.0018983944
## Tyrosine                                       Tyrosine 0.0112887118
## Uracil                                           Uracil 0.5429079298
## Valine                                           Valine 0.0001394238
## Xylose                                           Xylose 0.2154519963
## cis.Aconitate                             cis.Aconitate 0.0038978573
## myo.Inositol                               myo.Inositol 0.0022150347
## trans.Aconitate                         trans.Aconitate 0.0220856691
## pi.Methylhistidine                   pi.Methylhistidine 0.1413081041
## tau.Methylhistidine                 tau.Methylhistidine 0.0220399755
\end{verbatim}

\begin{Shaded}
\begin{Highlighting}[]
\CommentTok{\# Filtrar los metabolitos con valor p \textless{} 0.05}
\NormalTok{significant\_metabolites }\OtherTok{\textless{}{-}}\NormalTok{ results\_df[results\_df}\SpecialCharTok{$}\NormalTok{p\_value }\SpecialCharTok{\textless{}} \FloatTok{0.05}\NormalTok{, ]}
\end{Highlighting}
\end{Shaded}

\begin{Shaded}
\begin{Highlighting}[]
\CommentTok{\# Ver los metabolitos significativos}
\FunctionTok{head}\NormalTok{(significant\_metabolites)}
\end{Highlighting}
\end{Shaded}

\begin{verbatim}
##                                  metabolite     p_value
## X2.Aminobutyrate           X2.Aminobutyrate 0.027445476
## X2.Hydroxyisobutyrate X2.Hydroxyisobutyrate 0.005288021
## X3.Hydroxybutyrate       X3.Hydroxybutyrate 0.001185367
## X3.Hydroxyisovalerate X3.Hydroxyisovalerate 0.007827462
## X3.Indoxylsulfate         X3.Indoxylsulfate 0.008916134
## Acetate                             Acetate 0.006073195
\end{verbatim}

\begin{Shaded}
\begin{Highlighting}[]
\FunctionTok{nrow}\NormalTok{(significant\_metabolites)}
\end{Highlighting}
\end{Shaded}

\begin{verbatim}
## [1] 40
\end{verbatim}

En el análisis realizado, se identificaron numerosos metabolitos cuya
concentración se asocia de manera significativa con el estado de pérdida
muscular (Muscle.loss). Resulta evidente que, de forma previa a la
realización del análisis de muestras, todos los metabolitos objetivo son
sospechosos de tener relación directa con el estado de caquexia. Tal es
así que, de los 65 metabolitos analizados, hasta 40 de estos mostraron
una relación estadísticamente significativa (p \textless{} 0.05). Si
filtramos los metabolitos con un nivel de significancia más estricto (p
\textless{} 0.001), se obtienen 7 metabolitos altamente significativos,
lo que sugiere que estos compuestos podrían ser biomarcadores
potenciales para la caquexia.

\begin{Shaded}
\begin{Highlighting}[]
\CommentTok{\# Filtrar metabolitos con p \textless{} 0.01 (más estrictos)}
\NormalTok{highly\_significant\_metabolites }\OtherTok{\textless{}{-}}\NormalTok{ significant\_metabolites[significant\_metabolites}\SpecialCharTok{$}\NormalTok{p\_value }\SpecialCharTok{\textless{}} \FloatTok{0.001}\NormalTok{, ]}

\CommentTok{\# Mostrar los más relevantes}
\FunctionTok{print}\NormalTok{(highly\_significant\_metabolites)}
\end{Highlighting}
\end{Shaded}

\begin{verbatim}
##                              metabolite      p_value
## Creatinine                   Creatinine 0.0005129808
## Dimethylamine             Dimethylamine 0.0004460069
## Leucine                         Leucine 0.0002695046
## N.N.Dimethylglycine N.N.Dimethylglycine 0.0001554346
## Pyroglutamate             Pyroglutamate 0.0004845363
## Quinolinate                 Quinolinate 0.0001185108
## Valine                           Valine 0.0001394238
\end{verbatim}

\begin{Shaded}
\begin{Highlighting}[]
\FunctionTok{nrow}\NormalTok{(highly\_significant\_metabolites)  }\CommentTok{\# Número de metabolitos altamente significativos}
\end{Highlighting}
\end{Shaded}

\begin{verbatim}
## [1] 7
\end{verbatim}

Entre los metabolitos cuya concentración mostró una asociación más
significativa con el estado de pérdida muscular (p \textless{} 0.01), se
identificaron creatinina, dimethylamina, leucina, N.N.Dimethylglycina,
pyroglutamata, quinolinata y valina.

\subsection{5. Discusión}\label{discusiuxf3n}

En este estudio, se utilizó un set de datos sobre pacientes enfermos de
cáncer que presentaban o no caquexia. Para su análisis, se utilizó la
clase SummarizedExperiment con la finalidad de organizar y analizar los
datos con información sobre la expresión de metabolitos y los metadatos
asociados a las muestras de pacientes. En este caso,
SummarizedExperiment facilitó la integración de las mediciones de
metabolitos (almacenadas en el componente assays) con los metadatos
sobre condición de pérdida muscular (Muscle.loss) relacionados con los
pacientes (almacenados en colData). A diferencia de clases más antiguas
como ExpressionSet, SummarizedExperiment permitió trabajar con un solo
objeto que agrupaba todos los datos y metadatos relacionados,
simplificando su manejo y análisis. La estructura del objeto facilitó la
realización de análisis de datos multivariantes de manera eficiente, sin
necesidad de gestionar múltiples objetos en paralelo, lo que mejoró la
reproducibilidad y la organización del análisis.

El objetivo principal de esta actividad era la creación del objeto
SummarizedExperiment, su exploración y uso para análisis a elección. Por
ello se analizó la estructura del SE creado, encontrando que el SE
contiene una matriz de expresión accesible a través de
assay(se\_object), donde se visualizan las abundancias de los
metabolitos en cada paciente. Los metadatos de los pacientes,
almacenados en colData(se\_object), incluyen una columna clave llamada
Muscle.loss, que indica si el paciente presentaba caquexia o era parte
del grupo control, y los identificadores únicos de los pacientes,
accesibles con colnames(se\_object), permiten vincular los datos de
expresión con la información clínica correspondiente. Este análisis
preliminar confirmó la correcta estructura del SE, con los datos de
metabolómica y los metadatos clínicos organizados de manera coherente.
La dimensión del objeto (63 metabolitos x 77 pacientes) y la inspección
de los datos de expresión proporcionan una base sólida para el análisis
posterior.

Por último, se aplicó un análisis de varianza (ANOVA) para determinar la
relación entre los metabolitos y la condición de pérdida muscular de los
pacientes. Los resultados obtenidos mostraron que 40 de los 63
metabolitos analizados presentaron una relación estadísticamente
significativa con el estado de caquexia, con un valor de p \textless{}
0.05. Esto sugirió que estos metabolitos tienen una alta probabilidad de
tener un papel relevante en la patogénesis de la caquexia, por lo que su
análisis como biomarcadores potenciales en el diagnóstico o seguimiento
de esta condición es acertado por parte del estudio. Además, al filtrar
los metabolitos con un nivel de significancia más estricto (p
\textless{} 0.001), se identificaron 7 metabolitos altamente
significativos, lo que sugiere que estos compuestos podrían ser
biomarcadores esenciales. De hecho, algunos de estos metabolitos, como
creatinina, leucina y valina, han sido previamente asociados con la
caquexia y la pérdida muscular en estudios anteriores, lo que respalda
la validez de los resultados obtenidos (Okamura et al. 2023; Viana et
al. 2020).

\subsection{6. Conclusiones}\label{conclusiones}

La caquexia es una enfermedad compleja y multifactorial, caracterizada
por la pérdida de peso a través de la pérdida de masa muscular
esquelética y tejido adiposo, un desequilibrio en la regulación
metabólica y una reducción en la ingesta de alimentos. Las principales
causas de este trastorno son factores catabólicos producidos por tumores
en la circulación sistémica, así como factores fisiológicos como la
activación inflamatoria desequilibrada, la proteólisis, la autofagia y
la lipólisis que pueden ocurrir en cánceres gástrico, pancreático,
esofágico, pulmonar, hepático y colorrectal ({``{Caquexia y c{á}ncer -
Efectos secundarios}''} 2024).

En el presente estudio, la construcción y exploración del objeto
SummarizedExperiment permitieron organizar eficientemente los datos de
metabolómica y sus metadatos asociados al proyecto, asegurando una
estructura coherente para el análisis. A través del ANOVA, se
identificaron 40 metabolitos significativamente asociados con la
caquexia (p \textless{} 0.05), y 7 de ellos mostraron una relación
altamente significativa (p \textless{} 0.001), creatinina,
dimethylamina, leucina, N.N.Dimethylglycina, pyroglutamata, quinolinata
y valina. Estos hallazgos respaldan la relevancia de estos metabolitos
en la patogénesis de la caquexia y su potencial como biomarcadores clave
para el diagnóstico y seguimiento de la condición.

\subsection*{7. Bibliografía}\label{bibliografuxeda}
\addcontentsline{toc}{subsection}{7. Bibliografía}

\phantomsection\label{refs}
\begin{CSLReferences}{1}{0}
\bibitem[\citeproctext]{ref-CaquexiaCancerEfectos2024}
{``{Caquexia y c{á}ncer - Efectos secundarios}.''} 2024. \{cgvArticle\}.
https://www.cancer.gov/espanol/cancer/
tratamiento/efectos-secundarios/caquexia-cancer.

\bibitem[\citeproctext]{ref-ConvertSeuratObject}
{``Convert {Seurat Object} to {Summarised Experiment} or
{ExpressionSet}.''} n.d. https://support.bioconductor. org/p/9157595/.
Accessed April 1, 2025.

\bibitem[\citeproctext]{ref-cuiMetabolomicsItsApplications2022}
Cui, Pengfei, Xiaoyi Li, Caihua Huang, Qinxi Li, and Donghai Lin. 2022.
{``Metabolomics and Its {Applications} in {Cancer Cachexia}.''}
\emph{Frontiers in Molecular Biosciences} 9 (February): 789889.
\url{https://doi.org/10.3389/fmolb.2022.789889}.

\bibitem[\citeproctext]{ref-okamuraKidneyFunctionCachexia2023}
Okamura, Masatsugu, Masaaki Konishi, Javed Butler, Kamyar
Kalantar-Zadeh, Stephan von Haehling, and Stefan D. Anker. 2023.
{``Kidney Function in Cachexia and Sarcopenia: {Facts} and Numbers.''}
\emph{Journal of Cachexia, Sarcopenia and Muscle} 14 (4): 1589--95.
\url{https://doi.org/10.1002/jcsm.13260}.

\bibitem[\citeproctext]{ref-viana1HNMRBasedSerum2020}
Viana, Laís Rosa, Leisa Lopes-Aguiar, Rafaela Rossi Rosolen, Rogerio
Willians dos Santos, and Maria Cristina Cintra Gomes-Marcondes. 2020.
{``{1H-NMR Based Serum Metabolomics Identifies Different Profile}
Between {Sarcopenia} and {Cancer Cachexia} in {Ageing Walker} 256
{Tumour-Bearing Rats}.''} \emph{Metabolites} 10 (4): 161.
\url{https://doi.org/10.3390/metabo10040161}.

\end{CSLReferences}

\end{document}
